\documentclass[%
reprint,
superscriptaddress,
%groupedaddress,
%unsortedaddress,
%runinaddress,
%frontmatterverbose,
%preprint,
showpacs,preprintnumbers,
%nofootinbib,
%nobibnotes,
%bibnotes,
 amsmath,amssymb,
 aps,
%pra,
%prb,
prd,
%prl,
%rmp,
%prstab,
%prstper,
%floatfix,
]{revtex4-1}

\usepackage{float}
\usepackage{graphicx}% Include figure files
\usepackage{dcolumn}% Align table columns on decimal point
\usepackage{bm}% bold math
\usepackage{bbold}
\usepackage{amssymb,amsmath}
\usepackage{hyperref}% add hypertext capabilities
%\usepackage[mathlines]{lineno}% Enable numbering of text and display math
%\linenumbers\relax % Commence numbering lines

%\usepackage[showframe,%Uncomment any one of the following lines to test
%%scale=0.7, marginratio={1:1, 2:3}, ignoreall,% default settings
%%text={7in,10in},centering,
%%margin=1.5in,
%%total={6.5in,8.75in}, top=1.2in, left=0.9in, includefoot,
%%height=10in,a5paper,hmargin={3cm,0.8in},
%]{geometry}



\usepackage{color}
\usepackage{amsfonts}
\usepackage{subfigure}
\usepackage{array}


\newcommand{\Tr}{\ensuremath{\operatorname{Tr}}}
\newcommand{\tr}{\ensuremath{\operatorname{tr}}}
\newcommand{\Omegaqq}{\ensuremath{\Omega_{\bar{q}q}}}
\newcommand{\vev}[1]{\ensuremath{\left\langle #1 \right\rangle}}
\newcommand{\einh}[1]{\ensuremath{\,\text{#1}}}
\newcolumntype{L}{>{\centering\arraybackslash}m{3cm}}



\newcommand{\overbar}[1]{\mkern 1.5mu\overline{\mkern-1.5mu#1\mkern-1.5mu}\mkern 1.5mu}

\definecolor{bjcol}{rgb}{1,.44,0.13}

% color def's

\definecolor{blue}{rgb}{0,0,1}
\newcommand{\colb}[1]{{\color{blue} #1}}
\definecolor{green}{rgb}{0,1,0}
\newcommand{\colg}[1]{{\color{green} #1}}
\definecolor{red}{rgb}{1,0,0}
\newcommand{\colr}[1]{{\color{red} #1}}
\newcommand{\colJ}[1]{{\color{cyan} #1}}
\definecolor{gray}{rgb}{.5,.5,.5}
\newcommand{\drop}[1]{{\sout{ {\color{gray} #1}}}}
\definecolor{darkgreen}{rgb}{.0,.5,.0}
\newcommand{\colL}[1]{{\color{darkgreen} #1}}


\def\Fig#1{Fig.~\ref{#1}} \def\Tab#1{Tab.~\ref{#1}}
\def\Figs#1{Figs.~\ref{#1}} \def\Tab#1{Tab.~\ref{#1}}
\def\Eqs#1{Eqs.~(\ref{#1})}
\def\Eq#1{Eq.~(\ref{#1})}
\def\eq#1{(\ref{#1})}
\def\eqref#1{(\ref{#1})}
\def\fig#1{Fig.~\ref{#1}}
\def\tab#1{Tab.~\ref{#1}}
\def\eqs#1{(\ref{#1})}
\def\Eqs#1{(\ref{#1})}
\def\sec#1{Sec.~\ref{#1}}
\def\app#1{Appendix~\ref{#1}}
\newcommand{\Phibar}{\ensuremath{\bar{\Phi}}}
\newcommand{\LPQM}{\ensuremath{\mathcal{L}_{\textrm{PQM}}}\xspace}

\def\dbar{{\mathchar'26\mkern-12mu d}}
\def\lA0{{\langle A_0 \rangle}}
\def\bA0{{\bar{A}_0}}
\def\lLA{{\langle L[A_0] \rangle}}
\def\lL{{\langle L \rangle}}
\def\lLc{{\langle L^\dagger \rangle}}
\def\lLAc{{\langle L^\dagger[A_0] \rangle}}


\def\dr{{D\!\llap{/}}\,}
\def\Dr{{D\!\llap{/}}\,}
\def\ipv{\vec{p}\llap{/}}
\def\pslash{p\llap{/}}

\def\0#1#2{\frac{#1}{#2}}

\newcommand{\bsig}{\ensuremath{\bar{\sigma}}}
\newcommand{\lsm}{L\ensuremath{\sigma}M\xspace}
\newcommand{\pT}{\ensuremath{T_0}}
\newcommand{\Tl}{\ensuremath{T_\chi}}
\newcommand{\Ts}{\ensuremath{T_\chi^s}}
\newcommand{\Tchi}{\ensuremath{T_\chi}}
\newcommand{\Td}{\ensuremath{T_d}}
\newcommand{\Tc}{\ensuremath{T_c}}
\newcommand{\muc}{\ensuremath{\mu_c}}
\newcommand{\coloronl}{(color online)\xspace}

\newcommand{\mrm}[1]{\mathrm{#1}}
\def\qbar{\bar{q}}
\newcommand{\sx}{\sigma_{x}}
\newcommand{\sy}{\sigma_{y}}

%%%%%%%%%%%%%% for corrections %%%%%%%%%%%
\newcommand{\colsy}[1]{\textcolor{blue}{#1}}
\newcommand{\colrw}[1]{\textcolor{cyan}{#1}}
\newcommand{\colwjf}[1]{\textcolor{red}{#1}}

%
%%%%%%%%%%%%%%%%%%%%%%%%%%%%%%%%%%%%%%%%%%%%%%%%%%%%%%%%%%%%%%%%%%%%%%%%%%%%%

\graphicspath{{./figures/}{./}}

\begin{document}

\preprint{}

\title{Baryon number fluctuation and critical point
}

\author{Shi Yin}
\affiliation{School of Physics, Dalian University of Technology, Dalian, 116024,
  P.R. China}

\author{Rui Wen}
\affiliation{School of Physics, Dalian University of Technology, Dalian, 116024,
  P.R. China}

\author{Wei-jie Fu}
\email{wjfu@dlut.edu.cn}
\affiliation{School of Physics, Dalian University of Technology, Dalian, 116024,
  P.R. China}

%\date{\today}% It is always \today, today,
             %  but any date may be explicitly specified

\begin{abstract}

We investigate the relationship between the peak value of baryon number fluctuation kurtosis and the critical baryon chemical potential. At the same time, the freeze-out curves under different position of the critical end point. We control the position of the critical end point by include the fermion vacuum fluctuation gradually. This work is done under the low energy Polyakov-quark-meson model with the functional renormalization group approach.

\end{abstract}

%\pacs{Valid PACS appear here}% PACS, the Physics and Astronomy
\pacs{11.30.Rd, %Chiral symmetries
         11.10.Wx, %Finite-temperature field theory
         05.10.Cc, %Renormalization group methods
         12.38.Mh  %Quark-gluon plasma
     }                             % Classification Scheme.
%\keywords{Suggested keywords}%Use showkeys class option if keyword
                              %display desired
\maketitle

%\tableofcontents

%%%%%%%%%%%%%%%%%%%%%%%%%%%%%%%%%%%%%%%%%%%%%%%%%%%%%%%%%%%
%%%%%%%%%%%%%%%%%%%%%%%%%%%%%%%%%%%%%%%%%%%%%%%%%%%%%%%%%%%

\section{Introduction}
\label{sec:int}

The location of the critical end point (CEP) of the QCD phase diagram is a popular research direction in the field of high energy physics. However, the physical property at high baryon chemical potential is hard to study in both theoretical and experimental. In the experimental field, the Relativistic Heavy Ion Collider (RHIC) that provides us with a lot of experimental data \cite{Adamczyk:2013dal,Luo:2015ewa,Luo:2017faz}.





%%%%%%%%%%%%%%%%%%%%%%%%%%%%%%%%%%%%%%%%%%%%%%%%%%%%%%%%%%%%%
%%%%%%%%%%%%%%%%%%%%%%%%%%%%%%%%%%%%%%%%%%%%%%%%%%%%%%%%%%%%%

\section{}
\label{sec:FRG}




%%%%%%%%%%%%%%%%%%%%%%%%%%%%%%%%%%%%%%%%%%%%%%%%%%%%%%%%%%%%%
%%%%%%%%%%%%%%%%%%%%%%%%%%%%%%%%%%%%%%%%%%%%%%%%%%%%%%%%%%%%%

\section{}
\label{sec:EoS}

%
%%%%%%%%%%%%%%%%%%%%%%%%%%%%%
\begin{figure}[t]
\includegraphics[width=0.5\textwidth]{cp}
\caption{The relation between the max baryon number fluctuation kurtosis value and the baryon chemical potential of the critical end point. }\label{fig:cp}
\end{figure}
%%%%%%%%%%%%%%%%%%%%%%%%%%%%%
%

\begin{align}
   \chi_n^{B}&=\frac{\partial^n}{\partial (\mu_B/T)^n}\frac{p}{T^4}\,,\label{eq:suscept}
\end{align}
which is also called as the $n$-th order generalized susceptibility of the baryon number. $ \chi_n^{B}$'s in \Eq{eq:suscept} are related to the cumulants of the baryon number distributions, e.g.,
\begin{align}
  \chi_1^B&=\frac{1}{VT^3}\langle N_B \rangle\,,\\[2ex]
  \chi_2^B&=\frac{1}{VT^3}\langle(\delta N_B)^2\rangle\,,\\[2ex]
  \chi_3^B&=\frac{1}{VT^3}\langle(\delta N_B)^3\rangle\,,\\[2ex]
  \chi_4^B&=\frac{1}{VT^3}\Big(\langle(\delta N_B)^4\rangle-3\langle(\delta N_B)^2\rangle^2\Big)\,,
\end{align}



%%%%%%%%%%%%%%%%%%%%%%%%%%%%%%%%%%%%%%%%%%%%%%%%%%%%%%%%%%%%%
%%%%%%%%%%%%%%%%%%%%%%%%%%%%%%%%%%%%%%%%%%%%%%%%%%%%%%%%%%%%%

\section{}
\label{sec:num}


%%%%%%%%%%%%%%%%%%%%%%%%%%%%%%%%%%%%%%%%%%%%%%%%%%%%%%%%%%%%%
\subsection{}
\label{sec:1}





\section{Results and Summary}
\label{sec:res}





%%%%%%%%%%%%%%%%%%%%%%%%%%%%%%%%%%%%
\begin{acknowledgments}

The work was supported by the National Natural Science Foundation of China under Contracts Nos. 11775041.

\end{acknowledgments}


%%%%%%%%%%%%%%%%%%%%%%%%%%%%%%%%%%%%%%%%%%%%%%%%%%%%%%%%%%%%%
%%%%%%%%%%%%%%%%%%%%%%%%%%%%%%%%%%%%%%%%%%%%%%%%%%%%%%%%%%%%%

\appendix


%%%%%%%%%%%%%%%%%%%%%%%%%%%%%%%%%%%%%%%%%%%%




%%%%%%%%%%%%%%%%%%%%%%%%%%%%%%%%%%%%%%%%%%%%

% The \nocite command causes all entries in a bibliography to be
% printed out whether or not they are actually referenced in the
% text. This is appropriate for the sample file to show the different
% styles of references, but authors most likely will not want to use
% it.  \nocite{*}

%\bibliography{refspec}% Produces the bibliography via BibTeX.
\bibliography{ref-lib}% Produces the bibliography via BibTeX.


\end{document}
