\documentclass[%
reprint,
superscriptaddress,
%groupedaddress,
%unsortedaddress,
%runinaddress,
%frontmatterverbose,
%preprint,
showpacs,preprintnumbers,
%nofootinbib,
%nobibnotes,
%bibnotes,
 amsmath,amssymb,
 aps,
%pra,
%prb,
%prd,
prl,
%rmp,
%prstab,
%prstper,
%floatfix,
]{revtex4-1}

\usepackage{float}
\usepackage{graphicx}% Include figure files
\usepackage{dcolumn}% Align table columns on decimal point
\usepackage{bm}% bold math
\usepackage{bbold}
\usepackage{amssymb,amsmath}
\usepackage{hyperref}% add hypertext capabilities
%\usepackage[mathlines]{lineno}% Enable numbering of text and display math
%\linenumbers\relax % Commence numbering lines

%\usepackage[showframe,%Uncomment any one of the following lines to test
%%scale=0.7, marginratio={1:1, 2:3}, ignoreall,% default settings
%%text={7in,10in},centering,
%%margin=1.5in,
%%total={6.5in,8.75in}, top=1.2in, left=0.9in, includefoot,
%%height=10in,a5paper,hmargin={3cm,0.8in},
%]{geometry}



\usepackage{color}
\usepackage{amsfonts}
\usepackage{subfigure}
\usepackage{array}


\newcommand{\Tr}{\ensuremath{\operatorname{Tr}}}
\newcommand{\tr}{\ensuremath{\operatorname{tr}}}
\newcommand{\Omegaqq}{\ensuremath{\Omega_{\bar{q}q}}}
\newcommand{\vev}[1]{\ensuremath{\left\langle #1 \right\rangle}}
\newcommand{\einh}[1]{\ensuremath{\,\text{#1}}}
\newcolumntype{L}{>{\centering\arraybackslash}m{3cm}}



\newcommand{\overbar}[1]{\mkern 1.5mu\overline{\mkern-1.5mu#1\mkern-1.5mu}\mkern 1.5mu}

\definecolor{bjcol}{rgb}{1,.44,0.13}

% color def's

\definecolor{blue}{rgb}{0,0,1}
\newcommand{\colb}[1]{{\color{blue} #1}}
\definecolor{green}{rgb}{0,1,0}
\newcommand{\colg}[1]{{\color{green} #1}}
\definecolor{red}{rgb}{1,0,0}
\newcommand{\colr}[1]{{\color{red} #1}}
\newcommand{\colJ}[1]{{\color{cyan} #1}}
\definecolor{gray}{rgb}{.5,.5,.5}
\newcommand{\drop}[1]{{\sout{ {\color{gray} #1}}}}
\definecolor{darkgreen}{rgb}{.0,.5,.0}
\newcommand{\colL}[1]{{\color{darkgreen} #1}}


\def\Fig#1{Fig.~\ref{#1}} \def\Tab#1{Tab.~\ref{#1}}
\def\Figs#1{Figs.~\ref{#1}} \def\Tab#1{Tab.~\ref{#1}}
\def\Eqs#1{Eqs.~(\ref{#1})}
\def\Eq#1{Eq.~(\ref{#1})}
\def\eq#1{(\ref{#1})}
\def\eqref#1{(\ref{#1})}
\def\fig#1{Fig.~\ref{#1}}
\def\tab#1{Tab.~\ref{#1}}
\def\eqs#1{(\ref{#1})}
\def\Eqs#1{(\ref{#1})}
\def\sec#1{Sec.~\ref{#1}}
\def\app#1{Appendix~\ref{#1}}
\newcommand{\Phibar}{\ensuremath{\bar{\Phi}}}
\newcommand{\LPQM}{\ensuremath{\mathcal{L}_{\textrm{PQM}}}\xspace}

\def\dbar{{\mathchar'26\mkern-12mu d}}
\def\lA0{{\langle A_0 \rangle}}
\def\bA0{{\bar{A}_0}}
\def\lLA{{\langle L[A_0] \rangle}}
\def\lL{{\langle L \rangle}}
\def\lLc{{\langle L^\dagger \rangle}}
\def\lLAc{{\langle L^\dagger[A_0] \rangle}}


\def\dr{{D\!\llap{/}}\,}
\def\Dr{{D\!\llap{/}}\,}
\def\ipv{\vec{p}\llap{/}}
\def\pslash{p\llap{/}}

\def\0#1#2{\frac{#1}{#2}}

\newcommand{\bsig}{\ensuremath{\bar{\sigma}}}
\newcommand{\lsm}{L\ensuremath{\sigma}M\xspace}
\newcommand{\pT}{\ensuremath{T_0}}
\newcommand{\Tl}{\ensuremath{T_\chi}}
\newcommand{\Ts}{\ensuremath{T_\chi^s}}
\newcommand{\Tchi}{\ensuremath{T_\chi}}
\newcommand{\Td}{\ensuremath{T_d}}
\newcommand{\Tc}{\ensuremath{T_c}}
\newcommand{\muc}{\ensuremath{\mu_c}}
\newcommand{\coloronl}{(color online)\xspace}

\newcommand{\mrm}[1]{\mathrm{#1}}
\def\qbar{\bar{q}}
\newcommand{\sx}{\sigma_{x}}
\newcommand{\sy}{\sigma_{y}}

%%%%%%%%%%%%%% for corrections %%%%%%%%%%%
\newcommand{\colsy}[1]{\textcolor{blue}{#1}}
\newcommand{\colrw}[1]{\textcolor{cyan}{#1}}
\newcommand{\colwjf}[1]{\textcolor{red}{#1}}

%
%%%%%%%%%%%%%%%%%%%%%%%%%%%%%%%%%%%%%%%%%%%%%%%%%%%%%%%%%%%%%%%%%%%%%%%%%%%%%

\graphicspath{{./figures/}{./}}

\begin{document}

\preprint{}

\title{Baryon number fluctuation and critical end point
}

\author{Shi Yin}
\affiliation{School of Physics, Dalian University of Technology, Dalian, 116024,
  P.R. China}

\author{Rui Wen}
\affiliation{School of Physics, Dalian University of Technology, Dalian, 116024,
  P.R. China}

\author{Wei-jie Fu}
\email{wjfu@dlut.edu.cn}
\affiliation{School of Physics, Dalian University of Technology, Dalian, 116024,
  P.R. China}

%\date{\today}% It is always \today, today,
             %  but any date may be explicitly specified

\begin{abstract}

We investigate the relationship between the peak value of baryon number fluctuation kurtosis and the critical baryon chemical potential. At the same time, the freeze-out curves under different position of the critical end point. We control the position of the critical end point by a new cutoff scale which can remove the fermion vacuum fluctuation from the flow equations. This work is done under the low energy Polyakov-quark-meson model with the functional renormalization group approach.

\end{abstract}

%\pacs{Valid PACS appear here}% PACS, the Physics and Astronomy
\pacs{11.30.Rd, %Chiral symmetries
         11.10.Wx, %Finite-temperature field theory
         05.10.Cc, %Renormalization group methods
         12.38.Mh  %Quark-gluon plasma
     }                             % Classification Scheme.
%\keywords{Suggested keywords}%Use showkeys class option if keyword
                              %display desired
\maketitle

%\tableofcontents

%%%%%%%%%%%%%%%%%%%%%%%%%%%%%%%%%%%%%%%%%%%%%%%%%%%%%%%%%%%
%%%%%%%%%%%%%%%%%%%%%%%%%%%%%%%%%%%%%%%%%%%%%%%%%%%%%%%%%%%

\section{Introduction}
\label{sec:int}

The location of the critical end point (CEP) of the QCD phase diagram is a popular research direction in the field of high energy physics. However, the physical property at high baryon chemical potential that is the high density area is hard to study in both theoretical and experimental. In the experimental field, the Relativistic Heavy Ion Collider (RHIC) that provides us with a lot of experimental data at the high density part, see \cite{Adamczyk:2013dal,Luo:2015ewa,Luo:2017faz}.


%%%%%%%%%%%%%%%%%%%%%%%%%%%%%%%%%%%%%%%%%%%%%%%%%%%%%%%%%%%%%
%%%%%%%%%%%%%%%%%%%%%%%%%%%%%%%%%%%%%%%%%%%%%%%%%%%%%%%%%%%%%
\section{Baryon number fluctuations and the fermion vacuum fluctuation}
\label{sec:EoS}
The kurtosis of the baryon number fluctuation is the core of our work, for the significant role it plays in the experiment area of the QCD phase structure. The calculation of the baryon number fluctuation can be done using the following formula, 
\begin{align}
   \chi_n^{B}&=\frac{\partial^n}{\partial (\mu_B/T)^n}\frac{p}{T^4}\,,\label{eq:suscept}
\end{align}
in which the $\mu_B=3\mu$ is the beryon chemical potential that is triple of the quark chemical potential. Then we can obtain the first to fourth order of the beryon number fluctuation known as the generalized susceptibilities,
\begin{align}
  \chi_2^B&=\frac{1}{VT^3}\langle(\delta N_B)^2\rangle\,,\\[2ex]
  \chi_4^B&=\frac{1}{VT^3}\Big(\langle(\delta N_B)^4\rangle-3\langle(\delta N_B)^2\rangle^2\Big)\,,
\end{align}
The meaning of the angle brackets stands for the average value, and the $\delta N_B$ stands for the difference between the $N_B$ and $\langle N_B\rangle$ which reads $\delta N_B:=N_B - \langle N_B\rangle$.\par
For the purpose of comparing our calculation with the experimental results, we divide the fourth and second order of the baryon number fluctuations to get the kurtosis which is the observable in the experiments, e.g.,$\kappa\sigma^2=\chi^B_4/\chi^B_2$. More details discussion about baryon number fluctuation see \cite{Fu:2015naa}. Here we investigate the relationship of the maximum value of the kurtosis and the fermion vacuum fluctuation.
\par We find that the position of the critical end point would change with different fermion vacuum contribution we involve in our calculation. In this work we add a new cutoff scale into the functional renormalisation group flow equation to restrict the fermion vacuum fluctuation and study the behavior of the position of the critical end point under the different cutoff. From the previous work of the low energy effective theory under the FRG, the flow equation of the effective potential contain the full contribution of the fermion vacuum part throughout the integral interval which is cover from the infrared point to the ultraviolet point. The effect of the fermion vacuum contribution is studied in the mean field approximation, see \cite{Skokov:2010sf}. We can tell that the fermion vacuum fluctuation can suppress the baryon number fluctuation at finite temperature and density. If the new cutoff scale get the value of the UV scale, the flow return to the previous which includes all the fermion vacuum term. If the cutoff scale get the value of the IR scale, the flow get the result of the mean-field approximation. The neglect of the fermion vacuum part is known as the no-sea approximation.
%%%%%%%%%%%%%%%%%%%%%%%%%%%%%%%%%%%%%%%%%%%%%%%%%%%%%%%%%%%%%
%%%%%%%%%%%%%%%%%%%%%%%%%%%%%%%%%%%%%%%%%%%%%%%%%%%%%%%%%%%%%


\section{Polyakov-Quark-meson model and the cutoff}
\label{sec:pqm}




This work is done under the two quark flavor Polyakov-quark-meson model. We give the effective action

\begin{align}
  \Gamma_k=&\int_x \bigg\{Z_{q,k}\bar{q} \Big [\gamma_\mu \partial_\mu -\gamma_0(\hat\mu+igA_0) \Big ]q \nonumber\\[2ex]
  &+\frac{1}{2}Z_{\phi,k}(\partial_\mu \phi)^2\nonumber+h_k\bar{q}\big(T^0\sigma+i\gamma_5\vec{T}\cdot \vec{\pi}\big)q\\[2ex]
  &+V_k(\rho)-c\sigma \bigg\}\,,
\end{align}
with the 4 dimension integral $\int_x=\int^{1/T}_0dx_0\int d^3x$. The $k$ is the FRG infrared (IR) cutoff scale which is running from the ultraviolet (UV) scale to 0. The meson field is defined as $\phi=(\sigma,\vec{\pi})$, and $\rho=\phi^2/2$. $\vec{T}$ is the generators of the $SU(N_f)$ group here we have $N_f=2$. The generators satisfy $\mathrm{Tr}(T^iT^j)=\frac{1}{2}\delta^{ij}$, $T^0=\frac{1}{\sqrt{2N_f}}\mathbb{1}_{N_f\times N_f}$. The $-c\sigma$ gives the chiral symmetry breaking in our theory. In this work we get the results under the local potential approximation (LPA), under which $\partial_tZ_{\phi,q}=0$, $\partial_t h_k=0$. The $t$ is RG time with $t=\mathrm{ln}(k/\Lambda)$. We choose the UV cutoff scale $\Lambda$ to 700$\mathrm{MeV}$. The effective potential $V_k(\rho)$ involves the infermation of the meson chiral symmetry breaking. Here we solve the flow equation of the effective potential by the Taylor expansion around the expansion point $\kappa$. The expansed meson potential is $V_k(\rho)=\sum^{N_v}_{n=0}\frac{\lambda_{n,k}}{n!}(\rho-\kappa_k)^n$. In this work, we choose $N_v=5$ for the good convergence of the fix expansion point $\partial_t\kappa_k=0$. In order to get the pressure, the thermodynamical potential should be calculated. The definition of the thermodynamical potential is $\Omega[T,\mu]=V_{k=0}(\rho)+V_{glue}(L,\bar{L})-c\sigma$. The glue potential is a function of the traced Polyakov loop $L$ and the complex conjugate $\bar{L}$. They are in concerning with the gluonic background field $A_0$ by $L=\frac{1}{N_c}\langle \mathrm{Tr}\mathcal{P}\rangle$, $\bar{L}=\frac{1}{N_c}\langle \mathrm{Tr}\mathcal{P}^\dagger\rangle$ with $\mathcal{P}=\mathcal{P}\mathrm{exp}\big(ig\int^\beta_0d\tau A_0(\tau)\big)$. 



In the LPA situation, we focus on the flow of effective potential. The flow equation can be written as
\begin{align}
\begin{split}
   \partial_t V_k(\rho)=&\frac{k^4}{4\pi^2}\big[(N^2_f-1)l^{B,4}_{0}(\bar{m}^2_{\pi,k},\eta_{\phi,k};T)\\
   &+l^{(B,4)}_{0}(\bar{m}^2_{\sigma,k},\eta_{\phi,k};T)\\
   &-4N_cN_fl^{(F,4)}_{0}(\bar{m}^2_{q,k},\eta_{q,k};T,\mu)\big],
\end{split}
\end{align}
\par The anomalous dimensions of the fermion and boson is all 0 here. The $l^{(F,4)}_{0}$ in the flow equation stands for the fermion threshold function. The analytical form of the threshold function is 
\begin{align}
\begin{split}
l^{(F,d)}_{0}&(\bar{m}^2_{q,k},\eta_{q,k};T,\mu)=\frac{1}{z_q(d-1)\sqrt{1+\bar{m}^2_{q,k}}}(1-\frac{\eta_{q,k}}{d})\\
&\times(1-n_F(\bar{m}^2_{q,k},z_q;T,\mu)-n_F(\bar{m}^2_{q,k},z_q;T,\mu))
\end{split}
\end{align}
which contains all contribution of the fermion vacumm fluctuation. We take place this with another threshold function with no fermion vacuum part 
\begin{align}
\begin{split}
j^{(F,d)}_{0}&(\bar{m}^2_{q,k},\eta_{q,k};T,\mu)=\frac{1}{z_q(d-1)\sqrt{1+\bar{m}^2_{q,k}}}(1-\frac{\eta_{q,k}}{d})\\
&\times(-n_F(\bar{m}^2_{q,k},z_q;T,\mu)-n_F(\bar{m}^2_{q,k},z_q;T,\mu))
\end{split}
\end{align}
\par Then we devided the integral of the flow equation into two parts by the new cutoff scale $\Lambda_2$ between the IR and UV scales. The integral between the IR scale and $\Lambda_2$ is the flow equation with the fermion vacuum contribution the other side of the integral that between the $\Lambda_2$ and UV scale is the flow equation without the fermion vacuum contribution. Now we can control the quantity of the fermion vacuum part by changing the new cutoff $\Lambda_2$.
%%%%%%%%%%%%%%%%%%%%%%%%%%%%%%%%%%%%%%%%%%%%%%%%%%%%%%%%%%%%%
%%%%%%%%%%%%%%%%%%%%%%%%%%%%%%%%%%%%%%%%%%%%%%%%%%%%%%%%%%%%%
\section{Results and Summary}
\label{sec:res}
%
%%%%%%%%%%%%%%%%%%%%%%%%%%%%%
\begin{figure}[t]
\includegraphics[width=0.5\textwidth]{cp}
\caption{The relation between the maximum baryon number fluctuation kurtosis value and the baryon chemical potential of the critical end point. }\label{fig:cp}
\end{figure}
%%%%%%%%%%%%%%%%%%%%%%%%%%%%%
%
%
 %%%%%%%%%%%
\begin{table}[t]
  \centering
  \begin{tabular}{c||c|c|c|c|c|c|c|c}
    $\Lambda_2\,\,(\mathrm{MeV})$ & 0 & 100 & 200 & 300 & 400 & 500 & 600 & 700 \rule{0pt}{2.6ex}\rule[-1.2ex]{0pt}{0pt}\\ \hline\hline
    $T_c\,\,(\mathrm{MeV})$ &124& 123 & 117 &110 &98&97&95&89 \\\hline
    $\mu_Bc\,\,(\mathrm{MeV})$ &335 &345 &435 &552&655&725&770&785\\\hline

  \end{tabular}
  \caption{The position of the critical point and the value of the cutoff $\Lambda_2$.} 
  \label{tab:cut}
\end{table}
%%%%%%%%%%%%
%




%%%%%%%%%%%%%%%%%%%%%%%%%%%%%%%%%%%%
\section{acknowledgments}

The work was supported by the National Natural Science Foundation of China under Contracts Nos. 11775041.



%%%%%%%%%%%%%%%%%%%%%%%%%%%%%%%%%%%%%%%%%%%%

% The \nocite command causes all entries in a bibliography to be
% printed out whether or not they are actually referenced in the
% text. This is appropriate for the sample file to show the different
% styles of references, but authors most likely will not want to use
% it.  \nocite{*}

%\bibliography{refspec}% Produces the bibliography via BibTeX.
\bibliography{ref-lib}% Produces the bibliography via BibTeX.


\end{document}
